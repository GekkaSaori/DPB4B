%HEADER:
\documentclass{article}
\usepackage[UTF8]{ctex}%Chinese supporting
\usepackage{xcolor}%colors.
\usepackage{amsmath}%math fonts
\usepackage{amssymb}%math symbols
\usepackage{luatexja-ruby}%nihon kanji +rubi+
\usepackage[normalem]{ulem}%improved underlines
\usepackage{graphicx}%pictures
\usepackage[colorlinks,linkcolor=red]{hyperref}%hyperlinks
\usepackage{multirow}%form
\usepackage{multicol}%form
\usepackage{makecell}%form
\usepackage{minted}%code environment
\usepackage[amsmath,thmmarks]{ntheorem}%theorem environment
\usepackage{datetime2}
\usepackage{yhmath}
%DEFINE VERSION:


%Limitation:
% There MUST be a space after Proofs!!!
% Compile 3 Times Needed!!!
% Define fonts for nihon characters, and change the fonts for them.

% NEW FUNCTION DEFINATION:
\newcommand{\DeclareCJKFont}[2]{\setCJKfamilyfont{#1}{#2}}
% Argument #1: ShortName, Argument #2: FontName;
\newcommand{\CreateShortcutForCJKFont}[2]{\newcommand*{#1}[1]{{\CJKfamily{#2}##1}}}
% Argument #1: CommandName, Argument #2: ShortName;
% Example:
% 1. \DeclareCJKFont{stsong}{SimSun} SimSun was declared as stsong.
% 2. \CreateShortcut{\songtizi}{stsong} Use \songtizi{...} to typeset words in SimSun.
\newcommand{\mr}[2]{\multirow{#1}{*}{#2}}
% Define Shortcut for \multirow command.

% VERSION CONTROL:
% USE `ver.lua' script to print version.
\newcommand{\ver}{\directlua{dofile('./ver/ver.lua')}\alpha}
% DEFINE \bumprevsi TO BUMP THE REVISION VERSION:
\newcommand{\bumprevis}{\directlua{dofile('./ver/bumprevis.lua')}}
\newcommand{\bumpminor}{\directlua{dofile('./ver/bumpminor.lua')}}
\newcommand{\bumpmajor}{\directlua{dofile('./ver/bumpmajor.lua')}}
\newtheorem{solve}{解法}
% REDEFINE paragraph & subparagraph ENVIRONMENT:
\makeatletter
\renewcommand\paragraph{\@startsection{paragraph}{4}{\z@}%
{3.25ex \@plus1ex \@minus.2ex}%
{1.5ex \@plus.2ex}%
{\normalfont\normalsize\bfseries}}
\renewcommand\subparagraph{\@startsection{subparagraph}{5}{\z@}%
{3.25ex \@plus1ex \@minus.2ex}%
{1.5ex \@plus.2ex}%
{\normalfont\normalsize\bfseries}}
\makeatother

% NEW THEOREM ENVIRONMENTS DEFINATION:
\newtheorem{example}{例题}[subsection]
\newtheorem{definition}{定义}[subsection]
\newtheorem{thm}{定理}[subsection]
\theoremstyle{nonumberplain}
\theoremseparator{ }
\theoremheaderfont{\bf}
\theorembodyfont{\normalfont}
\theoremsymbol{\rule{1ex}{1ex}}
\newtheorem{Proof}{证明}

% NOTES:
\newcommand{\note}{\ \par
	\subsection*{课堂笔记\\\tiny{Note on the text}}
	\newpage}

% PRESETs for minted, Contents, CTeX, picutures folders:
\setminted[C++]{autogobble,breaklines,frame=lines,fontsize=\footnotesize,mathescape}
\setminted[text]
{autogobble,breaklines,frame=single,fontsize=\footnotesize,mathescape}
\ctexset{punct={kaiming},linestretch={2}}
\graphicspath{{./pictures/}}
\setcounter{secnumdepth}{4}
\setcounter{tocdepth}{4}

% FONTS DECLARATION:

%%%%%%%%%%%%%%%%%%%
