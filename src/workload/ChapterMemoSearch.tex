\section{记忆化搜索}
\subsection{我们为什么需要记忆化搜索?}
这一章我们讲记忆化搜索,记忆化搜索这个东西其实和普通的搜索没有太大的区别,是一个所谓“以空间换时间”的操作。

举个例子,我们要计算斐波那契数列的第$n$项。普通的递归可能会这么写:
\begin{minted}{C++}
int fib(int x)
{
    if(x==1||x==2)return 1;
    return fib(x-1)+fib(x-2);
}
\end{minted}

下图是当$n=8$时递归树的情况,可以看到,在调用过程中出现了大量的重叠子问题,而此时我们的程序不得不进行重复递归调用,这消耗了大量的时间。
\begin{center}
\begin{forest}
[8[7[6[5[4[3[2][1]][2]][3[2][1]]][4[3[2][1]][2]]][5[4[3[2][1]][2]][3[2][1]]]][6[5[4[3[2][1]][2]][3[2][1]]][4[3[2][1]][2]]]]
\end{forest}
\end{center}

再来更加极端一点的例子,下页图显示了当$n$的值扩大1时,问题的规模就几乎扩大了一倍。这几乎是不可接受的。

\begin{sidewaystable}
\begin{center}
\begin{forest}
[9[8[7[6[5[4[3[2][1]][2]][3[2][1]]][4[3[2][1]][2]]][5[4[3[2][1]][2]][3[2][1]]]][6[5[4[3[2][1]][2]][3[2][1]]][4[3[2][1]][2]]]][7[6[5[4[3[2][1]][2]][3[2][1]]][4[3[2][1]][2]]][5[4[3[2][1]][2]][3[2][1]]]]]
\end{forest}
\small{$n=9$\fangsong{时的递归树,可以看到重叠子问题更多了,问题规模几乎扩大了一倍。}}
\end{center}
\end{sidewaystable}
\begin{center}
\begin{tabular}{|c|c|}
\hline
$n$的取值&递归调用次数\\
\hline
$\cdots\cdots$&$\cdots\cdots$\\
\hline
40      &       204668309\\
\hline
41      &       331160281\\
\hline
42      &       535828591\\
\hline
43      &       866988873\\
\hline
44      &       1402817465\\
\hline
45      &       2269806339\\
\hline
$\cdots\cdots$&$\cdots\cdots$\\
\hline
\end{tabular}
\end{center}

上表展示了当$n$的取值较大时递归调用次数的变化情况,经模拟测算可以发现增长率几乎是指数级别的增长,事实证明也如此。

这样的时间复杂度几乎是不可接受的。所以我们需要使用记忆化搜索这一手段来优化。记忆化搜索,又称为带备忘的搜索,顾名思义,就是把已经搜索过的结果记录下来,作为一个“备忘”,当程序需要再次调用这个搜索过程时就直接调用这个结果即可,不需要重复搜索了。
\subsection{如何实现记忆化搜索?}
还是以刚才的斐波那契数列为例。我们可以开一个数组$f[]$,用于记录程序已经计算过的斐波那契数列的项的值。然后需要用的时候直接判断一下f数组里有没有这个元素(是否为初始值),如果有就直接返回这个值,没有再递归调用。代码如下:
\begin{minted}{C++}
int f[1000];//初始值为0,因为斐波那契数列中不可能出现0,所以0是非法的。
int fib(int x)
{
    if(x==1||x==2)return 1;//递归边界条件
    if(f[x]!=0)return f[x];//如果已经计算过了就返回结果。
    return f[x]=fib(x-1)+fib(x-2);//如果还没有计算过就递归计算并保存结果。
}
\end{minted}

下表展示了记忆化搜索递归调用的次数:
\begin{center}
\begin{tabular}{|c|c|}
\hline
$n$的取值&递归调用次数\\
\hline
$\cdots\cdots$&$\cdots\cdots$\\
\hline
40&3\\
\hline
41&3\\
\hline
42&3\\
\hline
43&3\\
\hline
44&3\\
\hline
45&3\\
\hline
$\cdots\cdots$&$\cdots\cdots$\\
\hline
\end{tabular}
\end{center}
可见,记忆化搜索能够帮我们减少递归的调用,提高代码的效率,以上代码即使在$n$非常大的情况下仍能较快的给出结果。

同时,不可忽略的一点是记忆化搜索需要有一个与状态等大的数组(当然你也可以用 \verb+map+或者\verb+set+ 之类的东西,只不过稍慢一些),空间占用很成问题。这也是记忆化搜索(包括DP)的弊端所在。当然由于DP在某些情况下可以优化空间复杂度(压维之类的),所以存在一类题目,可以使得DP通过而记忆化搜索MLE或TLE。

记忆化搜索通常适用于对于DFS的优化,对于BFS就不是那么实用了。当你在考场上写不出来DP但又非常确定这道题目是个DP的时候,不妨先打个暴力,然后转成记忆化搜索,可以帮助你拿到很多分数,还是非常实用的一种技巧。

下面给出普通搜索转记忆化搜索的模板:
\begin{minted}{C++}
int f[1000][1000];//与搜索的参数个数相同
void dfs(int a,int b)
{
    if(找到了这个问题的解) return ans;
    if(f[a][b]!=nil) return f[a][b];//这里的nil表示一个无效的值,可以认为是在正常情况下不可能出现的一个值
    //状态转移过程被省略一部分
    return f[a][b]=dfs(...,...);//这一句原来是return dfs(...,...);
}
\end{minted}
\note