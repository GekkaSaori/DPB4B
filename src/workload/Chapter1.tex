\section{引入------动态规划是什么?有什么用?能吃吗?}
\subsection{动态规划是什么?}
我们先来看一道题目:

\begin{example}由正实数构成的数字三角形排列如图所示,第一行的数为$a_{11}$;第二行的数从右到左依次为$a_{21},a_{22},\cdots$。第$n$行的数为$a_{n1},a_{n2},\cdots,a_{nn}$。从$a_{11}$开始,每一行的数$a_{ij}$只有两条边可以分别通向下一行的两个数$a_{(i+1)j}$和$a_{(i+1)(j+1)}$。用动态规划算法找出一条从$a_{11}$向下通到$a_{n1},a_{n2},\cdots,a_{nn}$中某个数的路径,使得该路径上的数之和达到最大。

	令$C[i,j]$是从$a_{11}$到$a_{ij}$路径上的数的最大和,并且$C[i,0]=C[0,j]=0$,则$C[i,j]=(\ \ \ )$。
	\begin{center}
		\begin{math}
			\begin{matrix}
				a_{11}                 \\
				a_{21}\ a_{22}         \\
				a_{31}\ a_{32}\ a_{33} \\
				\cdots\cdots           \\
				a_{n1}\ a_{n2}\ \cdots\cdots\cdots\cdots\ a_{nn}
			\end{matrix}
		\end{math}
	\end{center}

	\ \ A.\ $\max\{C[i-1,j-1],C[i-1,j]\}+a_{ij}$

	\ \ B.\ $C[i-1,j-1]+C[i-1,j]$

	\ \ C.\ $\max\{C[i-1,j-1,C[i-1,j]\}$

	\ \ D.\ $\max\{C[i,j-1],C[i-1,j]\}+a_{ij}$
\end{example}

这道题就是NOIp 2017提高组初赛试题第11题。就是一个经典的数字三角形问题。但是已经涉及到了动态规划最基本的内容了。动态规划最重要的两步就是设计状态和考虑状态转移方程。

以上内容涉及到两个非常重要的概念,状态和状态转移方程。我们先不学习DP,而是先温习一下我们已经具备的知识。

dalao们一定已经非常熟悉搜索的两种套路------BFS(广度优先搜索)与DFS(深度优先搜索)了。那我们就从我们已经会了的搜索开始讲。

实际上,DP就是一种无限优化了的搜索。本质与搜索并没有任何不同,那么为什么搜索的时间复杂度会达到$O(2^n)$,而DP可以做到$O(n^k)$的时间复杂度呢?前面说过,DP是一种优化的搜索。DP本质上是依靠状态的转移,来避免重复搜索的。这一点类似于记忆化搜索(搜索的优化),结果都是避免了重复搜索,从而提高了效率。不同之处在于,DP代码短(逃

下面给出两个概念的定义,希望大家能够理解:
\begin{definition}[状态]
	状态是指解决问题过程中的每个步骤或可能,类似搜索中的状态(还记得搜索题常见的cur和nxt这两个变量吗?就是这样的东西)。
\end{definition}
\begin{definition}[状态转移方程]
	状态转移方程是指解决问题过程中,由步骤$A$到步骤$B$所需要进行的操作以及状态发生的变化,类似搜索中状态的扩展(还记着八皇后问题中dx和dy这两个数组吗?就是类似这样的东西,实际上就是从上一步到这一步所需要进行的操作或者什么东西的改变)。
\end{definition}

回到刚才的例题。例题中的状态就是“$C[i,j]$是从$a_{11}$到$a_{ij}$的路径上的数的最大和”,转移过程题目中也给出来了,即“每一行的数$a_{ij}$只有两条边可以分别通向下一行的两个数$a_{(i+1)j}$和$a_{(i+1)(j+1)}$”,所以这一道题的$C[i,j]$由两个状态转移而来,即上一行的$C[i-1,j-1]$以及$C[i-1,j]$。在这两者之间取一个最大值,然后再加上$a_{ij}$这个数,就是转移到$C[i,j]$的过程。

所以说数字三角形的状态转移方程就是:
\begin{equation*}
	C[i,j]=\max\{C[i-1,j-1],C[i-1,j]\}+a_{ij}
\end{equation*}

这道题就做完了,选A。

通过这个例题,你应该对动态规划算法有了一个基础的了解,知道动态规划是个什么东西了。
\subsection{动态规划有什么用?能吃吗?}
这个问题问得好,动态规划没什么用,并且不能吃。(众人:那还学它做什么?!不是浪费时间吗?)

然而在NOIp中,动态规划通常会占0~200分,所以说学习DP还是有一些用处的。说它没用,在于一般是推不出来状态转移方程的,还不如打打暴力(像NOIp这种“暴力大赛”,打打暴力就可以拿到一等奖),并且NOIp给的暴力分也不少,这意味着我们学好暴力,每道DP题就可以拿到30~80分。

当然这是对于我们蒟蒻而言,dalao们肯定都会正解\sout{,那我就不讲了}。

不可否认,DP作用很大,但是也可以用别的方式来替代,虽然不能拿到满分,但是也可以得到相对比较可观的分数,所以学习DP是建立在练习好暴力的基础之上。这样可以保证你在NOIp赛场上不至于有一道或者多道题目爆零(这点很重要,不要瞧不起这5分10分,积少成多,聚沙成塔,部分分多的话也是很可观的)。

当然DP也有它的局限性,包括但不限于:
\begin{itemize}
	\item{解决问题不能具有后效性(当前状态不能影响之前的状态);}
	\item{解决的问题类型必须是\textbf{\underline{最优化问题}};}
	\item{内存占用多(相对于DFS而言)。}
\end{itemize}

\begin{center}\includegraphics[scale=0.60]{38183202_p0.jpg}\end{center}
\note

\section*{\textcolor{white}{?????}}
\begin{quote}

	\textcolor{white}{いつだって\ruby{君}{きみ}は\ruby{嗤}{わら}われ\ruby{者}{もの}だ\\
		你一直以来都是被嘲笑的人}

	\textcolor{white}{やることなすことツイてなくて\\
		不论作什么事情都无法顺利完成}

	\textcolor{white}{\ruby{挙句}{あげく}に\ruby{雨}{あめ}に\ruby{降}{ふら}られ\\
		最终陷入雨中}

	\textcolor{white}{お\ruby{気}{き}にの\ruby{傘}{かさ}は\ruby{風}{かぜ}で\ruby{飛}{と}んでって\\
		喜欢的雨伞都被风吹走}

	\textcolor{white}{そこのノラはご\ruby{苦}{く}\ruby{労様}{ろうさま}と\\
		路边的流浪汉一边对你说声辛苦了}

	\textcolor{white}{\ruby{足}{あし}を\ruby{踏}{ふ}んづけてった\\
		一边踩过你的雨伞}

	\ \par

	\textcolor{white}{いつもどおり\ruby{君}{きみ}は\ruby{嫌}{きら}われものだ\\
		一如往常般的你是个被讨厌的人}

	\textcolor{white}{\ruby{何}{なに}にもせずとも\ruby{遠}{とお}ざけられて\\
		就算什么事也没做却依然的被排挤在外}

	\textcolor{white}{\ruby{努}{ど}\ruby{力}{りょく}をしてみるけど\\
		虽然试着去努力过溶入人群}

	\textcolor{white}{その\ruby{理}{り}\ruby{由}{よう}なんて「なんとなく?」で\\
		但得到的理由竟是「不知不觉就忘了你」}

	\textcolor{white}{\ruby{君}{きみ}は\ruby{途}{と}\ruby{方}{ほう}に\ruby{暮}{く}れて\ruby{悲}{かな}しんでた\\
		你无计可施并开始感到悲伤}
	\begin{flushright}\textcolor{white}{------}\em{\textcolor{white}{``Odds \& Ends"}}\end{flushright}
\end{quote}

\textcolor{white}{希望大家不要放弃,OIer这个群体很是特殊,并且饱受排挤,大家一定坚持下去,不管外界怎么看待我们,一定要记住自己的目标。不忘初心,方得始终。}
\newpage