\section{对动态规划的初步了解}
动态规划通常是用来处理一些最优化问题, 需要问题具有可以划分阶段的特性以及其抽象模型所需要满足的一些条件.
可以将对问题一定阶段的解法抽象成一个状态, 从这一阶段的状态得到下一阶段的状态的过程被称为状态转移, 我们的目标就是求出完整问题的最优状态.
\subsection{一些定义}
\begin{definition}[动态规划(dynamic programming)]是运筹学的一个分支,是求解决策过程(decision process)最优化的数学方法。20世纪50年代初美国数学家R.E.Bellman等人在研究多阶段决策过程(multistep decision process)的优化问题时,提出了著名的最优化原理(principle of optimality),把多阶段过程转化为一系列单阶段问题,利用各阶段之间的关系,逐个求解,创立了解决这类过程优化问题的新方法——动态规划。
\end{definition}
\begin{definition}[状态]
	状态是指解决问题过程中的每个步骤或可能,类似搜索中的状态(还记得搜索题常见的cur和nxt这两个变量吗?就是这样的东西)。
\end{definition}
\begin{definition}[状态转移方程]
	状态转移方程是指解决问题过程中,由步骤$A$到步骤$B$所需要进行的操作以及状态发生的变化,类似搜索中状态的扩展(还记着八皇后问题中dx和dy这两个数组吗?就是类似这样的东西,实际上就是从上一步到这一步所需要进行的操作或者什么东西的改变)。
\end{definition}

\subsection{动态规划实例}
我们先来看一道题目:

\begin{example}由正实数构成的数字三角形排列如图所示,第一行的数为$a_{11}$;第二行的数从右到左依次为$a_{21},a_{22},\cdots$.第$n$行的数为$a_{n1},a_{n2},\cdots,a_{nn}$。从$a_{11}$开始,每一行的数$a_{ij}$只有两条边可以分别通向下一行的两个数$a_{(i+1)j}$和$a_{(i+1)(j+1)}$。用动态规划算法找出一条从$a_{11}$向下通到$a_{n1},a_{n2},\cdots,a_{nn}$中某个数的路径,使得该路径上的数之和达到最大.
	
	令$f(i,j)$是从$a_{11}$到$a_{ij}$路径上的数的最大和,并且$f(i,0)=f(0,j)=0$,则$f(i,j)=(\ \ \ )$.
	\begin{center}
		\begin{math}
		\begin{matrix}
		&a_{11}&         \\
		&a_{21}&\ a_{22}         \\
		&a_{31}&\ a_{32}&\ a_{33} \\
		&\vdots&\vdots&\vdots&       \\
		&a_{n1}&\ a_{n2}&\ \cdots&\ a_{nn}
		\end{matrix}
		\end{math}
	\end{center}
	
	\ \ A.\ $\max_\{f(i-1,j-1),f(i-1,j)\}+a_{ij}$
	
	\ \ B.\ $f(i-1,j-1)+f(i-1,j)$
	
	\ \ C.\ $\max\{f(i-1,j-1),f(i-1,j)\}$
	
	\ \ D.\ $\max\{f(i,j-1),f(i-1,j)\}+a_{ij}$
\end{example}

题目中涉及的$f(i,j)$其实就是一个可行的\textbf{状态设计}, 状态设计所必须满足的条件是它包含着所有需要的信息, 对例题来说只需要知道此到达位置的\textbf{最大收益}而无需知道\textbf{具体路径}.

这道题就是NOIp 2017提高组初赛试题第11题,一个经典的数字三角形问题。但是已经涉及到了动态规划的本质了。动态规划最重要的两步就是\textbf{设计状态}和考虑\textbf{状态转移方程}。
以上内容涉及到两个非常重要的概念,状态和状态转移方程。我们先不学习DP,而是先温习一下我们已经\textbf{具备的知识}。

同学们大概是会做基本的递推和搜索了.

大部分动态规划都可以被表示成一种递推的形式.

对于搜索,fDP就可以看成是一种搜索。本质与搜索并没有任何不同,那么为什么搜索的时间复杂度会达到$O(2^n)$,而DP可以做到$O(n^k)$的时间复杂度呢?

大家应该知道,搜索是利用状态转移来完成问题的整个步骤的,动态规划也是这么做的,我认为它与搜索的最本质区别是它是利用上一步的最优状态来进行这一步的转移.而搜索是利用上一步的所有状态来进行本步的转移并且在其中选择最优的那个来更新答案.

这就可以看出,动态规划能做的事情搜索都是能做的.但是有搜索能做但是动态规划不能做的.

什么时候动态规划能解决这个问题呢?问题关键是"最优"两个字.
这里会涉及到做动态规划的两个前提: \textbf{最优子结构} 和 \textbf{无后效性} .


\subsubsection{最优子结构}尝试从直观上去体会它的含义.

对于一个可以划分为若干阶段的问题.如果能利用它的一个\textbf{子问题的最优解}得到\textbf{整个问题的最优解}.那么这个问题是具有最优子结构的.

对于这个问题,从起点出发到达终点的最优解是整个问题的解,而从起点到达其中不是终点的某个点的最优解是整个问题一个子问题的解.

考虑这个例题具不具有最优子结构.
我们可以发现,如果最终的最优解出现在$f(u,v)$,也就是从起点以某条路径走到$f(u,v)$会获得最大收益.

考虑到达$(u,v)$的路径,我们知道只有两个点能到达$(u,v)$,也就是$(u-1,v-1)$和$(u-1,v)$,那么容易知道问题的最优解一定是从这两个点的最优解转移过来的.不然一定会有更优的解.
说明这道例题是具有最优子结构的.
所以通过求出$f(u,v-1)$和$f(u-1,v-1)$就可以得到$f(u,v)$.




\subsubsection{无后效性}
无后效性指的是问题的一个阶段的状态已知,那么这个阶段此后的阶段只与此阶段有关,而与之前发生过的所有阶段的状态无关,也不会影响到未来的任何阶段.
对于例题,从起点到达某个点$(u,v)$的最大收益需要通过$f(u-1,v)$和$f(u-1,v-1)$来求解,而不需要知道$f(u-2,v-1)$或者再之前的状态.状态$f(u,v)$也不会影响到$f(u-2,v-1)$等未来的状态.
我们知道例题同样是满足无后效性的.

通过对问题的进一步学习我们会知道对于一个问题的无后效性取决于\textbf{状态设计}.


\begin{solve}
	通过对问题的进一步分析容易得到问题的解法.
	其状态转移方程可以这么写:
	\begin{equation*}
	f(i,j)=\max\{f(i-1,j),f(i-1,j-1)\}+a_{ij}
	\end{equation*}
	根据在递推方面的经验我们可以如果需要求出最终结果$f(u,v)$.
	我们首先需要求出$f(u-1,v-1), f(u-1,v)$ .
	如果一步一步逆推的话, 大概需要递归来解决.
	我们大概需要在程序中设计一个函数$f(i,j)$.
	\begin{minted}{C++}
	f(i,j)
	return max{f(i-1,j),f(i-1,j-1)}+a(i,j)
	
	ans()
	return f(u,v)
	\end{minted}
	而对于这个问题而言正推和逆推并没有本质区别.
	
	按照递推的设计思路, 这个程序应该这样设计.
	\begin{minted}{C++}
	for(int i=1;i<=u;++i)
	for(int j=1;j<=v;++j)
	f[i][j]=max(f[i-1][j]+f[i-1][j-1])+a[i][j];
	\end{minted}
\end{solve}
\subsection{动态规划的应用及其局限性}
大家应该提高对于动态规划的重视.
在NOIp中,动态规划通常会出1~2个题目, 当然并不是所有人都能全部做出来的, 说它没用,是因为如果不能想出可靠的转移, 反而浪费了大量的时间, 还不如写最简单的算法, 并且NOIp通常情况下会给不少的部分分,这部分部分分通常用不到60行代码就能拿到, 这意味着我们学好暴力,每道DP题就可以拿到20~60分。

动态规划当然不是万能的, 有很大的\textbf{局限性}, 这一点在前面有所介绍.
\begin{itemize}
	\item{所解决的问题必须是满足\textbf{最优子结构}和\textbf{无后效性}的\textbf{可分阶段}的问题.}
	\item{某些能用动态规划解决的问题可能会有其他更好更优的算法.}
	\item{其本身的局限性, $n^k$的时间复杂度以及同样规模的空间占用.}
	\item{设计出足够优秀的状态这有时是很难的, 见多识广很重要.}
\end{itemize}
\newpage
\begin{example}\textbf{最大正方形}\\
	\textbf{题目描述}
	
	在一个$n\times m$的只包含$0$和$1$的矩阵里找出一个不包含$0$的最大正方形,输出边长。\\
	\textbf{输入格式}
	
	输入文件第一行为两个整数$n,m(1\leq n,m\leq 100)$,接下来$n$行,每行$m$个数字,用空格隔开,$0$或$1$.\\
	\textbf{输出格式}
	
	一个整数,最大正方形的边长\\
	\textbf{输入样例}
	
	\begin{minted}{text}
	4 4
	0 1 1 1
	1 1 1 0
	0 1 1 0
	1 1 0 1
	\end{minted}
	\textbf{输出样例}
	\begin{minted}{text}
	2
	\end{minted}
	
\end{example}

这个题目和上一个题目是比较像的, 在状态设计上和转移上.只需要对题目进行合适的建模.

考虑如下的状态设计, $f(i,j)$表示以$(i,j)$为右下角元素的最大正方形的大小.容易发现它满足最优子结构和无后效性.
\begin{center}
	\begin{math}
	\begin{matrix}
	\ddots& \vdots& \vdots& \vdots& \vdots& \vdots& \adots& \\ 
	\cdots& 1     & 1     & 1     & \cdots& 1     & \cdots& \\
	\cdots& 1     & 1     & 1     & \cdots& 1     & \cdots& \\
	\cdots& \vdots& \vdots& \vdots& \ddots& \vdots& \cdots& \\
	\cdots& 1     & 1     & 1     & \cdots& 1     & \cdots& \\
	\cdots& 1     & 1     & 1     & \cdots& 1     & \cdots& \\ 
	\adots& \vdots& \vdots& \vdots& \vdots& \vdots& \ddots& 
	\end{matrix}
	\end{math}
\end{center}
\par 我们容易发现如果对于一个位置$(i,j)$, 以它为右下角元素存在一个边长大于$2$的正方形, 那么$(i-1,j-1),(i-1,j),(i,j-1)$都必定存在以其为右下角元素的正方形.
考虑状态$f(i,j)$由哪些状态转移而来, 容易发现, \begin{center}\textbf{如果$f(i,j)=k$, 那么$f(i,j-1),f(i-1,j-1),f(i-1,j)\geq k-1$}\end{center}


\note

\textcolor{white}{希望大家不要放弃,OIer这个群体很是特殊,并且饱受排挤,大家一定坚持下去,不管外界怎么看待我们,一定要记住自己的目标。不忘初心,方得始终。}
\newpage